\begin{frame}{Future}
\begin{block}{Comunicación asíncrona con Future}
\begin{itemize}
  \item \#include \textless future\textgreater .
  \item La función \textbf{async} nos permite lanzar la ejecución de una hebra sin que necesariamente comience su ejecución inmediatamente.
  \item Si la función que lanzamos asíncronamente retornar algún valor, podemos consultar por el valor retornado almacenándolo en una variable del tipo \textbf{future}.
  \item La clase \textit{future} es una clase \textit{template} (de igual forma que \textit{atomic}), en donde se debe especificar el tipo de dato que será utilizado en el \textit{template}. El tipo de dato tiene que corresponderse con el tipo de dato retornado por la función lanzada asíncronamente.
\end{itemize}
\end{block}
\end{frame}

\begin{frame}{Future}
\begin{block}{Comunicación asíncrona con Future}
\begin{itemize}
  \item En general se puede ver el \textit{header future} como un conjunto de utilidades que permiten acceso asíncrono a valores que serán seteados por \textit{proveedores} corriendo en otras hebras de ejecución.
  \item El método \textbf{get} es el método principal de los objetos del tipo \textit{future}. El método \textit{get} es el que obliga a que el valor sea finalmente calculado/seteado en caso en que todavía no lo haya sido. En el caso en que el valor todavía no se encuentre disponible, la hebra ejecutando el método \textit{get} se \textbf{bloqueará} hasta que el valor se encuentre disponible.
  \item La hebra calculando el valor puede setear una \textbf{excepción} en el valor a retornar, lanzándose efectivamente al momento en que se realiza la llamada a \textit{get}.
\end{itemize}
\end{block}
\end{frame}

\begin{frame}[fragile]{Future}
\begin{block}{Ejemplo de future}
\begin{lstlisting}[language=C++, basicstyle=\small]
#include <future>

int add(int a, int b) {
  return a+b;
}

int main(int argc, char *argv[]) {
  std::future<int> result ( std::async(add, 2, 4) );
  
  std::cout << result.get() << std::endl;
  
  return 0;
}
\end{lstlisting}
\end{block}
\end{frame}

\begin{frame}{Future}
\begin{block}{Proveedores para future}
\begin{itemize}
  \item La función \textbf{async} es la más fácil de utilizar para proveer de valores a las variables del tipo future.
  \item Además de la función \textit{async}, se cuenta con los objetos de las clases \textit{promise} y \textit{packaged\_task}.
  \item La clase \textbf{promise} es una clase \textit{template} que permite proveer de \textbf{valores} o de \textbf{excepciones} a variables del tipo \textit{future}.
  \item Los métodos principales de un objeto promise son \textbf{get\_future}, \textbf{set\_value} y \textbf{set\_exception}.
\end{itemize}
\end{block}
\end{frame}

\begin{frame}{Future}
\begin{block}{Proveedores para future - promise}
\begin{itemize}
  \item El método \textbf{get\_future} es la forma en la que se \textbf{vincula} el \textit{future} con el \textit{promise}.
  \item Se debe invocar al método \textit{get\_future} cuando se está construyendo/incializando el objeto del tipo \textit{future}. Posterior a esta invocación, el objeto del tipo \textit{promise} y \textit{future} poseerán un estado compartido.
  \item Solo puede retornarse \textbf{un future por promise}.
  \item Una vez que se ha generado el vínculo, se espera que en algún momento el \textit{promise} setee el valor esperado por el \textit{future}, o que simplemente setee alguna excepción para que sea lanzada por el \textit{future} al intentar recuperar el valor con el método \textit{get}.
\end{itemize}
\end{block}
\end{frame}

\begin{frame}{Future}
\begin{block}{Proveedores para future - promise}
\begin{itemize}
  \item Si se llegase a ejecutar el destructor de la variable de tipo \textit{promise} sin que ésta haya alcanzado a setear algún valor o excepción, el estado de la información pasará a estar como \textbf{disponible} y la invocación del método \textit{get} del \textit{future} retornará. En el momento en que retorne \textit{get}, se lanzará una excepción del tipo \textbf{future\_error}.
  \item Cuando en alguna hebra de ejecución se invoca a una de los métodos \textbf{set\_value} o \textbf{set\_exception}, el estado cambia a disponible y la hebra bloqueada en la invocación a \textit{get} retorna.
\end{itemize}
\end{block}
\end{frame}

\begin{frame}[fragile]{Future}
\begin{block}{Ejemplo de future y promise}
\begin{lstlisting}[language=C++, basicstyle=\small]
#include <future>
#include <thread>
#include <exception>

void print(std::future<int> & future) {
  try {
    std::cout << future.get() << std::endl;
  } catch(std::exception & e) {
    std::cerr << e.what() << std::endl;
  }
}
\end{lstlisting}
\end{block}
\end{frame}

\begin{frame}[fragile]{Future}
\begin{block}{Ejemplo de future y promise}
\begin{lstlisting}[language=C++, basicstyle=\small]
int main(int argc, char *argv[]) {
  std::promise<int> promise;
  std::future<int> future ( promise.get_future() );
  std::thread thread (print, std::ref(future));
  promise.set_value(15);
  
  thread.join();
  
  return 0;
}
\end{lstlisting}
\end{block}
\end{frame}

\begin{frame}{Future}
\begin{block}{Proveedores para future - packaged\_task}
\begin{itemize}
  \item La clase \textbf{packaged\_task} funciona de forma similar al \textit{promise} pero se diferencian en que la primera envuelve a un \textbf{objeto llamable} (una función, una expresión lambda, un functor etc.).
  \item El valor retornado por el objeto funcional es lo que se setea como valor a almacenar en el \textit{future}, es decir, ya no es necesario especificar el valor a través de alguna llamada a \textit{set\_value} como se hacía en los objetos \textit{promise}.
  \item La clase packaged\_task es una clase template y dentro de los parámetros del template se debe especificar el tipo de retorno de la función y los tipos de variables de los parámetros.
\end{itemize}
\end{block}
\end{frame}

\begin{frame}[fragile]{Future}
\begin{block}{Ejemplo de future y packaged\_task}
\begin{lstlisting}[language=C++, basicstyle=\small]
int add(int a, int b) {
  return a+b;
}

int main(int argc, char *argv[]) {
  std::packaged_task<int(int,int)> task (add);
  std::future<int> future ( task.get_future() );
  std::thread thread (std::move(task), 3, 4);
  std::cout << future.get() << std::endl;
  thread.join();  
  return 0;
}
\end{lstlisting}
\end{block}
\end{frame}