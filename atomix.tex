\begin{frame}{Atomic}
\begin{block}{Características principales}
\begin{itemize}
  \item \#include \textless atomic\textgreater .
  \item \textbf{Atomic} permite la implementación de algoritmos libres de \textit{locks}.
  \item Operaciones atómicas presentan mejor performance que su equivalente utilizando locks.
  \item Uno debe especificar el tipo de variable que será tratada de forma atómica (\textit{thread-safe}).
  \item Va a depender de la implementación el mecanismo utilizado para asegurar que la operación sea \textit{thread-safe}, pudiendo incluso llegar a utilizarse \textit{locks}. Generalmente para \textbf{tipos de datos básicos}, las implementaciones quedan \textbf{libres de locks}.
\end{itemize}
\end{block}
\end{frame}

\begin{frame}{Atomic}
\begin{block}{Características principales}
\begin{itemize}
  \item Los métodos principales son load y store.
  \item El método \textbf{load} funciona como los métodos \textit{get} y simplemente retorna el valor actual. El método \textbf{store} setea un nuevo valor a la variable del tipo \textit{atomic}.
  \item Existe otro método llamado \textbf{exchange}, el cual setea un nuevo valor a la variable atómica y al mismo tiempo retorna el valor antiguo.
\end{itemize}
\end{block}
\end{frame}

\begin{frame}[fragile]{Atomic}
\begin{block}{Ejemplo de atomic}
\begin{lstlisting}[language=C++, basicstyle=\small]
#include <atomic>
int main(int argc, char *argv[]) {
  int initialValue = 5;
  std::atomic<int> atomic (initialValue);
  int newValue = 10;
  atomic.store(newValue);
  int lastValue = atomic.load();
  
  return 0;
}
\end{lstlisting}
\end{block}
\end{frame}

\begin{frame}{Atomic}
\begin{block}{Características principales}
\begin{itemize}
  \item Los métodos \textit{load} y \textit{store} permiten además especificar el \textbf{modelo de memoria} a utilizar (secuencialmente consistente es el modelo por defecto).
  \item Los modelos de memoria describen si los accesos a memoria pueden ser optimizados ya sea por el compilador o por el procesador. Si el modelo es relajado, se permite que las operaciones sean reordenadas o combinadas con el fin de mejorar el performance.
  \item El modelo más estricto es el secuencialmente consistente, el cual es el por defecto.
\end{itemize}
\end{block}
\end{frame}

\begin{frame}[fragile]{Atomic}
\begin{block}{Ejemplo del método exchange}
\begin{lstlisting}[language=C++, basicstyle=\small]
#include <atomic>
std::atomic<bool> winner (false);

void race() {
  if (!winner.exchange(true)) {
    std::cout << "winner id = ";
    std::cout << std::this_thread::get_id() << std::endl;
  }
}
\end{lstlisting}
\end{block}
\end{frame}

\begin{frame}[fragile]{Atomic}
\begin{block}{Ejemplo del método exchange}
\begin{lstlisting}[language=C++, basicstyle=\small]
#include <thread>

int main(int argc, char *argv[]) {
  std::thread threads[10];
  
  for (unsigned i = 0; i < 10; i++) {
    threads[i] = std::thread(race);
  }
  
  for (unsigned i = 0; i < 10; i++) {
    threads[i].join();
  }
  
  return 0;
}
\end{lstlisting}
\end{block}
\end{frame}