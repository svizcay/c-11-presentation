\begin{frame}{Mutex}
\begin{block}{Exclusión mutua - Mutex}
\begin{itemize}
  \item Para hacer uso de exclusión mutua, se debe incluir el header \textless mutex\textgreater .
  \item Ejemplo de uso: deseamos obtener una salida limpia en \textit{cout}, es decir, que no se mezcle la escritura de hebras concurrentes.
\end{itemize}
\end{block}
\end{frame}

\begin{frame}[fragile]{Mutex}
\begin{block}{Ejemplo básico}
\begin{lstlisting}[language=C++, basicstyle=\small]
#include <mutex>
void print(int value1, int value2, int value3) {
  static std::mutex m;
  m.lock();
  std::cout << value1 << " ";
  std::cout << value2 << " ";
  std::cout << value3 << " ";
  std::cout << std::endl;
  m.unlock();
}
\end{lstlisting}
\end{block}
\end{frame}

\begin{frame}{Mutex}
\begin{block}{Analizando el ejemplo anterior}
\begin{itemize}
  \item Cada vez que una hebra intenta invocar la función \textit{print}, esta intentará adquirir el \textbf{lock} del \textit{mutex}.
  \item Una vez que una hebra adquiere el \textit{lock}, ésta entra a lo que es conocido como \textbf{sección crítica}. La sección crítica en este caso son las líneas de código que imprimen los 3 valores en la pantalla.
  \item Ninguna otra hebra puede entrar a la sección crítica mientras el \textit{lock} se encuentre tomado.
  \item Una vez que la hebra que adquirió el lock termina de ejecutar la sección crítica, ésta libera el lock manualmente realizando un \textbf{unlock}.
\end{itemize}
\end{block}
\end{frame}

\begin{frame}{Mutex}
\begin{block}{Analizando el ejemplo anterior}
\begin{itemize}
  \item \textbf{NOTA:} la variable mutex, la cual provee los métodos lock y unlock, es declarada \textbf{static} con el fin de que se cree solo una variable del tipo mutex \textbf{común para todas las invocaciones} a \textit{print}.
  \item Cuando se hace uso de \textit{locks} se serializa el código. \textbf{Atomix} es una alternativa al uso de \textit{locks} pero solo algunos tipos de variables permiten ser manejados a nivel atómico, además de ser solo algunas operaciones las que efectivamente se permiten que se realicen de forma atómica.
  \item Notar que si llega ocurrir un error dentro de la sección crítica, la hebra que se encuentre ejecutando la sección crítica \textbf{nunca alcanzará a ejecutar el \textit{unlock}} del \textit{mutex}, produciendo que la aplicación se cuelge completamente.
\end{itemize}
\end{block}
\end{frame}

\begin{frame}[fragile]{Mutex}
\begin{block}{Ejemplo 2: Alternativa try-catch}
\begin{lstlisting}[language=C++, basicstyle=\small]
#include <mutex>
#include <exception>
void print(int value1, int value2, int value3) {
  static std::mutex m;
  m.lock();
  try {
    // critical section
    m.unlock();
  } catch(std::exception &e) {
    m.unlock();
  }
}
\end{lstlisting}
\end{block}
\end{frame}

\begin{frame}{Mutex}
\begin{block}{Mutex con RAII}
\begin{itemize}
  \item Una alternativa mejor sería que el \textit{lock} se liberase \textbf{automáticamente} una vez que \textbf{finaliza el scope del mutex}. Así el programador no tendría que preocuparse de liberar manualmente el \textit{lock} poniéndose en cada escenario en que el bloque de código de la sección crítica pudiese fallar.
  \item Dentro de C++11 se sugiere el uso de manejadores de locks. \textbf{RAII} (Resource Acquisition Is Initialization) es un término que se refiere a que la adquisición de un recurso está limitada por el tiempo de vida  del objeto que lo adquiere. Este idea también puede ser encontrada como \textbf{CADRe} (Constructor Acquires, Destructor Release).
  \item RAII provee encapsulación y seguridad en caso que ocurran excepciones.
\end{itemize}
\end{block}
\end{frame}


\begin{frame}[fragile]{Mutex}
\begin{block}{Ejemplo 3: Uso de lock\_guard como manejador de locks}
\begin{lstlisting}[language=C++, basicstyle=\small]
#include <mutex>
void print(int value1, int value2, int value3) {
  static std::mutex m;
  std::lock_guard<std::mutex> lock(m);
  // critical section
}
\end{lstlisting}
\end{block}
\end{frame}

\begin{frame}{Mutex}
\begin{block}{Usando lock\_guard}
\begin{itemize}
  \item El constructor de lock\_guard invoca al método \textit{lock} del \textit{mutex}.
  \item En el instante en que se acaba el \textit{scope} del objeto \textit{lock\_guard} o alguna excepción es arrojada, el destructor del \textit{lock\_guard} realiza una llamada al método \textit{unlock} del \textit{mutex}.
\end{itemize}
\end{block}
\end{frame}

\begin{frame}{Mutex}
\begin{block}{Usando unique\_lock}
\begin{itemize}
  \item Existe otro manejador de mutex aparte del \textit{lock\_guard}, llamado \textbf{unique\_lock}. La diferencia entre \textit{unique\_lock} y \textit{guard\_lock}, es que el último, solo permite la invocación del \textit{lock} del \textit{mutex} al momento de construcción del objeto \textit{guard\_lock}, es decir, el manejador \textit{unique\_lock} es \textbf{más flexible} permitiendo mayor control sobre la invocación de los métodos \textit{lock} y \textit{unlock}.
  \item Un objeto del tipo \textit{unique\_lock} permite ser contruido sin necesidad de llamar al método \textit{lock} del \textit{mutex} interno. De todas formas se asegura que al momento en que se acaba su \textit{scope}, el \textit{mutex} se encontrará liberado.
\end{itemize}
\end{block}
\end{frame}

\begin{frame}{Mutex}
\begin{block}{Usando unique\_lock}
\begin{itemize}
  \item Cuando se utilizan \textbf{variables de condición} es necesario tener que liberar el \textit{lock} antes de que la hebra se bloquee al tener que ejecutar el \textit{wait}. El uso de \textit{guard\_lock} resulta ser no suficiente para tales casos debido a que se deben especificar \textit{locks} y \textit{unlocks} adicionales a los momento en que se construye y destruye el manejador de \textit{locks}.
  \item Como regla, \textbf{solo} cuando se realicen \textbf{llamados explícitos} al método \textit{lock} del \textit{mutex} se deberá también llamar explícitamente al método \textit{unlock}.
\end{itemize}
\end{block}
\end{frame}

\begin{frame}{Mutex}
\begin{block}{Tipos de mutex}
\begin{description}
  \item[Mutex:] El \textit{mutex} tradicional. Si una misma hebra llama al método \textit{lock} de un \textit{mutex} dos veces consecutivas sin hacer un \textit{unlock}, la segunda llamada producirá un \textbf{deadlock}.
  \item[Recursive\_mutex:] Permite que el \textit{lock} sea obtenido más de una vez por la misma hebra (similar a la característica de exclusión mutua en $\mu$C++).
\end{description}
\end{block}
\end{frame}

\begin{frame}{Mutex}
\begin{block}{Tipos de mutex}
\begin{description}
  \item[Timed\_mutex:] \textit{Mutex} que además de los métodos \textit{lock} y \textit{unlock}, proveen también los métodos \textbf{try\_lock\_for} y \textbf{try\_lock\_until}, los cuales intentan durante una determinada cantidad de tiempo adquirir el \textit{lock}. Ambos métodos retornan después de tal periodo independientemente de si lograron o no adquirirlo. En caso en que la adquisición fue existosa, los métodos retornan verdadero y cierran el \textit{lock} (\textbf{recordar hacer el posterior unlock}). En caso que expire el \textit{timeout}, los métodos retornan falso y no hay nada que se necesite liberar.
  \item[Recursive\_timed\_mutex:] Es la unión de las dos características anteriores.
\end{description}
\end{block}
\end{frame}

\begin{frame}{Mutex}
\begin{block}{Spinlock con try\_lock}
\begin{itemize}
  \item El método \textbf{try\_lock} es común a todos los tipos de \textit{mutex}. Este método puede ser utilizado para hacer un \textbf{spinlock}.
  \item Recordar que cuando una hebra encuentra que el \textit{lock} está tomado, ésta se \textbf{bloquea} esperando a que se libere el \textit{lock}. Un \textit{spinlock}, en vez de bloquear a la hebra, comienza a hacer lo que es conocido como \textbf{busy-waiting}.
\end{itemize}
\end{block}
\end{frame}

\begin{frame}[fragile]{Mutex}
\begin{block}{Spinlock con try\_lock}
\begin{lstlisting}[language=C++, basicstyle=\small]
#include <mutex>
int main(int argc, char *argv[]) {
  std::mutex mutex;
  while (!mutex.try_lock()) {
    ;	// busy-waiting
  }
  // critical section
  mutex.unlock();
  
  return 0;
}

\end{lstlisting}
\end{block}
\end{frame}

\begin{frame}{Mutex}
\begin{table}[h]\footnotesize
\centering
\caption{Resumen tipo de mutex y métodos}
\label{my-label}
\begin{tabular}{|l|c|c|c|c|}
\hline
\rowcolor[HTML]{ECF4FF} 
\multicolumn{1}{|c|}{\cellcolor[HTML]{ECF4FF}{\bf Método}} & {\bf Mutex} & {\bf Recursive\_mutex} & {\bf Timed\_mutex} & {\bf Recursive\_timed\_mutex} \\ \hline
\rowcolor[HTML]{EFEFEF} 
lock                                                       & X           & X                      & X                  & X                             \\ \hline
\rowcolor[HTML]{EFEFEF} 
unlock                                                     & X           & X                      & X                  & X                             \\ \hline
\rowcolor[HTML]{EFEFEF} 
try\_lock                                                  & X           & X                      & X                  & X                             \\ \hline
\rowcolor[HTML]{CBCEFB} 
try\_lock\_for                                             &             &                        & X                  & X                             \\ \hline
\rowcolor[HTML]{CBCEFB} 
try\_lock\_until                                           &             &                        & X                  & X                             \\ \hline
\end{tabular}
\end{table}
\end{frame}

\begin{frame}{Mutex}
\begin{table}[h]\footnotesize
\centering
\caption{Resumen tipo de manejadores de mutex y métodos}
\label{my-label}
\begin{tabular}{|l|c|c|}
\hline
\rowcolor[HTML]{ECF4FF} 
\multicolumn{1}{|c|}{\cellcolor[HTML]{ECF4FF}{\bf Método}} & {\bf Lock\_guard} & {\bf Unique\_lock} \\ \hline
\rowcolor[HTML]{EFEFEF} 
lock                                                       &                   & X                  \\ \hline
\rowcolor[HTML]{EFEFEF} 
unlock                                                     &                   & X                  \\ \hline
\rowcolor[HTML]{EFEFEF} 
try\_lock                                                  &                   & X                  \\ \hline
\rowcolor[HTML]{CBCEFB} 
try\_lock\_for                                             &                   & X                  \\ \hline
\rowcolor[HTML]{CBCEFB} 
try\_lock\_until                                           &                   & X                  \\ \hline
\end{tabular}
\end{table}
\begin{itemize}
  \item \textit{Lock\_guard} obviamente no posee ningún método dado que el \textit{lock} y \textit{unlock} lo realiza al instante en que se construye y destruye el objeto. \textit{Unique\_lock} posee todos los métodos pero para invocar a los dos últimos, el \textit{mutex} manejado debe ser del tipo \textit{timed}.
\end{itemize}
\end{frame}
%==================================================================================
\begin{frame}[fragile]{Mutex}
\begin{block}{Simulando un monitor de $\mu$C++}
\begin{lstlisting}[language=C++, basicstyle=\small]
#include <mutex>

class Monitor {
public:
  Monitor();
  ~Monitor();
  
private:
  std::recursive_mutex mutex;
  
  void method1();
  void method2();  
};
\end{lstlisting}
\end{block}
\end{frame}

\begin{frame}[fragile]{Mutex}
\begin{block}{Simulando un monitor de $\mu$C++}
\begin{lstlisting}[language=C++, basicstyle=\small]
#include "monitor.hpp"

void Monitor::method1() {
  std::unique_lock<std::recursive_mutex> lock (mutex);
  // critical section
}

void Monitor::method2() {
  std::unique_lock<std::recursive_mutex> lock (mutex);
  // critical section
}
\end{lstlisting}
\end{block}
\end{frame}


%==================================================================================
\begin{frame}{Mutex}
\begin{block}{Funciones adicionales}
\begin{itemize}
  \item Dentro del header \textit{mutex} se pueden encontrar las siguientes funciones adicionales: \textbf{lock}, \textbf{try\_lock}, \textbf{call\_once}.
  \item La \textbf{función lock} toma como parámetro una lista de objetos del tipo \textit{mutex} de los cuales intentará obtener el \textit{lock}. En caso en que no se logren obtener todos los \textit{locks}, la función realizará un \textbf{rollback} quedando todos los \textit{locks} nuevamente liberados. En caso de éxito, la función retornará y la hebra podrá acceder a la sección crítica. Una vez que se ejecutó la sección crítica, se deben liberar de forma manual todos los \textit{locks} que se obtuvieron.
\end{itemize}
\end{block}
\end{frame}

\begin{frame}[fragile]{Mutex}
\begin{block}{Uso de la función lock}
\begin{itemize}
  \item El \textbf{orden} en que se especifican los mutex \textbf{no es importante}. La función fue diseñada de tal forma para aliviar al programador el tener que escribir la adquisición de \textit{locks} en un determinado orden con el fin evitar \textit{deadlocks}.
\end{itemize}
  \begin{columns}
  \begin{column}{.45\textwidth}
\begin{lstlisting}[language=C++, basicstyle=\small]
#include <mutex>
std::mutex mutex1, mutex2;
void task1() {
  std::lock(mutex1, mutex2);
  // critical section
  mutex1.unlock();
  mutex2.unlock();
}
\end{lstlisting}
  \end{column}
  \begin{column}{.45\textwidth}
\begin{lstlisting}[language=C++, basicstyle=\small]


void task2() {
  std::lock(mutex2, mutex1);
  // critical section
  mutex1.unlock();
  mutex2.unlock();
}
\end{lstlisting}
  \end{column}
  \end{columns}
\end{block}
\end{frame}

\begin{frame}{Mutex}
\begin{block}{Funciones adicionales}
\begin{itemize}
  \item La función \textbf{try\_lock} funciona de forma similar a la función \textit{lock} pero invocará al método \textit{try\_lock} de todos los \textit{mutex} que se le indiquen dentro de la lista de parámetros.
  \item \textit{Try\_lock} retornará -1 en caso en que obtenga todos los \textit{locks} satisfatoriamente o un número entero indicando el índice del primer \textit{mutex} del cual no logró adquirir su \textit{lock}.
\end{itemize}
\end{block}
\end{frame}

\begin{frame}{Mutex}
\begin{block}{Funciones adicionales: call\_once y once\_flag}
\begin{itemize}
  \item Si tenemos una función que será ejecutada por múltiples hebras y deseamos que cierta parte de la función sea solo ejecutada por tan solo una hebra, podemos utilizar la función \textbf{call\_once} para especificar las líneas de código que deben ejecutarse solo una vez.
  \item La función call\_once recibe como parámetro una función, un objeto funcional, un lambda o cualquer otro objeto que se comporte como función para indicar el código a ejecutar.
  \item \textit{Call\_once} necesita verificar una variable del tipo \textbf{once\_flag} para ver si el código ya fue o no ejecutado por alguna otra hebra.
\end{itemize}
\end{block}
\end{frame}

\begin{frame}[fragile]{Mutex}
\begin{block}{Uso de call\_once}
\begin{lstlisting}[language=C++, basicstyle=\small]
#include <mutex>

std::once_flag flag;

void task() {
  std::call_once(flag, []() {
    std::cout << "print just once" << std::endl;
  });
  std::cout << "print every single time" << std::endl;
}

\end{lstlisting}
\end{block}
\end{frame}